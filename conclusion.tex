\section{Заключение}

В ходе выполнения курсовой работы были получены следующие результаты:

\begin{itemize}
 \item Изучены особенности работы с инструментами для анализа обменов на шинах данных;
 \item Проанализирована архитектура существующего решения для анализа обменов на бортовых авиационных шинах данных Opermon;
 \item На базе Opermon реализован инструмент для анализа обменов на шине I2C с использованием адаптера Bus Pirate.
\end{itemize}

Предложения по дальнейшей работе:

\begin{itemize}
 \item Добавление в реализацию специальных форматов сообщений с учётом адреса устройства (например, для широко распространённых микросхем и сенсоров с интерфейсом I2C);
 \item Добавление поддержки работы с другими интерфейсами, поддерживаемыми Bus Pirate (например, SPI или 1-Wire);
 \item Реализация режимов работы Opermon как ведущего или ведомого устройства для активного анализа устройств или для имитационного моделирования во встраиваемых системах.
\end{itemize}

