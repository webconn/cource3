\section{Заключение}

В ходе выполнения курсовой работы были получены следующие результаты:

\begin{itemize}
 \item проведён анализ существующих на рынке решений для анализа обменов на шинах данных, используемых во встраиваемых системах;
 \item выполнено сравнение этих решений с разрабатываемой в ЛВК средой Opermon;
 \item предложен и реализован проект инструмента для анализа обменов на шине I2C с использованием адаптера Bus Pirate;
 \item предложен проект рефакторинга существующей архитектуры среды Opermon с целью отделения интерфейсо-специфичных компонентов.
\end{itemize}

Предложения по дальнейшей работе:

\begin{itemize}
 \item добавление в реализацию специальных форматов сообщений с учётом адреса устройства (например, для широко распространённых микросхем и сенсоров с интерфейсом I2C);
 \item добавление поддержки работы с другими интерфейсами, поддерживаемыми Bus Pirate (например, SPI или 1-Wire);
 \item реализация режимов работы Opermon как ведущего или ведомого устройства для активного анализа устройств или для имитационного моделирования во встраиваемых системах;
 \item реализация проекта рефакторинга и подготовка дистрибутива Opermon для опубликования в открытом доступе.
\end{itemize}

