\section{Обзор инструментальной среды Opermon}

Opermon - инструмент для отображения, анализа и сбора трасс, созданный изначально для работы с бортовыми авиационными шинами данных, такими как MSTD-1553, ARINC, FibreChannel и CAN. На сегодняшний день Opermon имеет следующие возможности:

\begin{itemize}
 \item отображение обменов в табличном виде;
 \item сохранение трасс для последующего анализа (с возможностью удаления старых обменов из трассы для длительной непрерывной работы);
 \item разбор сообщений с выделением параметров;
 \item отображение значений параметров в табличном виде или в виде графиков.
\end{itemize}

Решение Opermon имеет клиент-серверную архитектуру, где разделены инструмент взаимодействия с адаптером и средство визуализации. (Структурная схема решения представлена в приложении~\ref{app:figures} на рисунке~\ref{fig:opermon_base}).

\subsection{Серверная часть}

Задача удалённого агента - устанавливать соединение с адаптером и обеспечивать обмен данными между адаптером и средством визуализации Opermon. Взаимодействие с Opermon происходит через пару TCP-соединений.

Первое соединение служит для передачи команд управления через RPC (Remote Procedure Call, вызов удалённых процедур) с помощью библиотеки qxmlrpc [\ref{qxmlrpc}]. Агент ожидает подключения средства визуализации.

Второе TCP-соединение открывается по команде bind от средства визуализации. Через это соединение передаются данные о прочитанных обменах, а также сообщения в очередь на передачу через адаптер (если реализация интерфейса в Opermon поддерживает передачу данных на шину).

\subsection{Клиентская часть}

В задачи средства визуализации входит предоставление пользовательского интерфейса для подключения к агенту (или нескольким агентам) и для настройки параметров адаптера.

