\section{Анализ существующих решений}

\label{i2c_review}

На сегодняшний день разработчикам встраиваемых систем доступен довольно широкий спектр различных программно-аппаратных комплексов для регистрации и анализа обменов на шине I2C. Для подготовки собственного решения есть смысл ознакомиться с ними, провести сравнительный анализ и выделить слабые и сильные стороны каждого из них.

\subsection{Преобразователи I2C $\leftrightarrow$ TTY}

Одним из самых простых решений для прослушивания шины является подключение к ней через простой по устройству адаптер, имеющий последовательный интерфейс, Разработчику доступно множество вариантов, начиная от несложных самодельных адаптеров, использующих микроконтроллер общего назначения в качестве преобразователя интерфейса, до серийно выпускаемых универсальных адаптеров.

С представителями первой группы можно ознакомиться по ссылкам: [\ref{arduino-i2c-sniffer}], [\ref{i2c-spi-1w-debugger}].

Вторая группа представлена следующими адаптерами: [\ref{buspirate_descr}].

Зачастую эти простые преобразователи подразумевают их использование вместе со специальным программным обеспечением, предоставляя помимо консольного интерфейса бинарный вариант.

\subsubsection*{Плюсы}

\begin{itemize}

 \item \textbf{Простота исполнения и дешевизна}. Как правило, при проектировании подобных адаптеров используются недорогие распространённые компоненты, при этом зачастую допуская изготовление платы в "кустарных" условиях. Как следствие, такой адаптер может позволить себе любой заинтересованный разработчик.

 \item \textbf{Отсутствие необходимости в специфическом ПО}. Большинство адаптеров этого типа имеют текстовый консольный интерфейс, поэтому для начала работы с ними требуется только программа-терминал последовательного порта.

 \item \textbf{Возможность интеграции с пользовательским ПО}. Как правило, программный интерфейс таких адаптеров открыт, достаточно прост и не использует специфических решений для обмена данными с ПК (работа с последовательным портом доступна практически на всех платформах с использованием почти всех распространённых языков программирования). Таким образом, у пользователя есть возможность интегрировать взаимодействие с адаптером в своём собственном ПО.
 
Можно также заметить, что при использовании подобных адаптеров при определённой сноровке есть возможность записывать трассы обменов для последующего анализа, так как данные представляются в виде простого текстового (или бинарного) потока. Более того, записанные данные (в случае текстового потока) можно пытаться анализировать даже без использования специального ПО. Также есть возможность преобразовать записанные данные в требуемый пользователю формат (например, XML или CSV) с помощью несложных скриптов.

\end{itemize}

\subsubsection*{Минусы}

\textit{Замечание: рассматриваются отрицательные стороны использования решения в консольном режиме, без специальных утилит.}

\begin{itemize}
 \item \textbf{Представление данных}. Чаще всего, консольный интерфейс сильно ограничен в плане представления данных (так как по своей природе может выводить только текст). При большом количестве обменов на линии такой интерфейс становится чрезвычайно неудобным.
 
 \item \textbf{Фильтрация передач}. Как правило, простые аппаратные анализаторы не имеют возможности предоставить выборку данных по шаблону (как и пытаться декодировать посылку стандартного формата).
\end{itemize}

\subsubsection*{Выводы}

Решения, основанные на использовании простых преобразователей с консольным интерфейсом неплохо подходят для анализа редко возникающих событий на шине с небольшими объёмами передаваемых данных. Они дешевы и в общем случае не требуют специального (возможно, дорогостоящего) ПО для начала работы. Более того, за счёт относительной простоты и открытости интерфейса взаимодействия с адаптером, они оставляют пользователю определённую свободу в выборе инструмента анализа, начиная от ПО для графического представления и анализа данных (при преобразовании данных трасс в поддерживаемые форматы, такие как CSV или XML) и заканчивая специальными скриптами для разбора обменов.

\subsection{Специализированные программно-аппаратные комплексы}

Отдельного внимания заслуживают специализированные программно-аппаратные решения для визуализации и анализа обменов, в том числе коммерческие закрытые решения. При проектировании собственного решения очень полезно ознакомиться с опытом уже существующих, сравнить их с точки зрения предлагаемых возможностей, удобства в использовании, а также оценить сильные и слабые стороны каждого из них.

\subsubsection*{Beagle I2C/SPI Protocol Analyzer + Data Center}

Beagle I2C/SPI Protocol Analyzer [\ref{beagle_protocol_analyzer}] - устройство для захвата и анализа, разработанное американской компанией Total Phase, специализирующаейся на решениях для разработки и отладки встраиваемых систем. Оно подключается к ПК посредством интерфейса USB 2.0 и использует собственное API для взаимодействия с приложениями.

Анализатор способен считывать данные с линии I2C, работающей на скорости до 4 Мбод (спецификация I2C High-speed mode), при этом работает исключительно как сниффер (пассивный анализатор обменов).

Для работы с линейкой анализаторов от Total Phase разработано программное обеспечение Data Center [\ref{tp_data_center}]. По типу представления данных ПО идеологически близко к WireShark.

Пользовательский интерфейс инструмента Data Center включает в себя следующие базовые элементы:

\begin{itemize}
 \item \textbf{Таблица обменов}. Отображает полученные обмены в расшифрованном виде (с учётом типа интерфейса и протокола обмена).
 \item \textbf{Сырые данные обмена}. Представляет данные в битовом виде (16-ричные значения полученных данных).
 \item \textbf{Командная строка}. Позволяет управлять функционалом ПО с помощью текстовых команд.
 \item \textbf{Информация об устройстве}. Отображает список подключенных к ПК устройств, доступных для анализа, а также подробную информацию о текущем выбранном устройстве.
\end{itemize}

Снимок экрана с рабочим окном Data Center представлен в приложении~\ref{app:figures} на рисунке~\ref{fig:datacenter}.

\subsubsection*{Aardwark Host Adapter + Control Center}

Помимо линейки анализаторов шин данных, компания Total Phase предлагает устройства и ПО для активного вмешательства в работу шины. Это необходимо, например, для получения данных в шине без master-устройства (либо где роль master-устройства играет ПК с подключенным адаптером), либо для эмуляции оконечных устройств шины средствами ПК с адаптером.

Для работы с шиной I2C предлагается адаптер Aardwark I2C/SPI Host Adapter [\ref{aardwark_adapter}] и ПО Control Center [\ref{tp_control_center}]. Адаптер подключается к ПК посредством интерфейса USB 2.0 и использует собственное API для взаимодействия с пользовательскими приложениями.

ПО Control Center позволяет ПК выступать в роли master- или slave-устройства на шине I2C или SPI.

Возможности ПО Control Center:

\begin{itemize}
 \item \textbf{Настройка напряжения логических уровней и питания целевого устройства}. Адаптер Aardwark I2C/SPI Host Adapter имеет встроенный конвертер логических уровней и программируемый источник питания, от которого можно передавать питание отлаживаемому устройству. Настройка этих блоков производится из пользовательского интерфейса Control Center.
 \item \textbf{Ручное взаимодействие с шиной}. Пользовательский интерфейс Control Center включает в себя виджеты для ручного управления шиной в разных режимах: генерирование запросов в режиме master, приём или передача данных в режимах master или slave.
 \item \textbf{Ведение лога}. В лог вносятся события о получении или передаче данных по шине, а также об изменениях настроек конвертеров уровней и источников питания.
\end{itemize}

Снимок экрана с рабочим окном Control Center представлен в приложении~\ref{app:figures} на рисунке~\ref{fig:controlcenter}.

\subsubsection*{Логические анализаторы BitScope Logic}

Отдельной категорией устройств для отладки и мониторинга шин данных являются универсальные логические анализаторы. Эти устройства имеют несколько (как правило, порядка 16) входных каналов, способных отслеживать логические уровни на линиях данных. Логические анализаторы обычно подключаются к ПК посредством USB или через более скоростную шину PCI.

Программное обеспечение для работы с логическими анализаторами обычно ориентировано на отображение текущего состояния входных каналов подобно осциллографу, в виде графика зависимости уровня от времени. Такие анализаторы чрезвычайно полезны для поиска ошибок на физическом уровне.

Интересным представителем ПО для работы с логическими анализаторами является BitScope Logic [\ref{bitscope_logic}]. Оно используется для работы с оборудованием BitScope.

Возможности BitScope Logic:
\begin{itemize}
 \item \textbf{Отображение состояния входов логического анализатора}. На экран выводится график зависимости уровня на входе от времени.
 \item \textbf{Расшифровка обменов}. BitScope Logic поддерживает работу с интерфейсами UART, I2C, SPI, CAN и многими другими и имеет возможность расшифровывать (представлять в текстовом виде) обмены этих интерфейсов.
 \item \textbf{Представление данных в табличном виде}. В ПО есть возможность представления обменов в традиционном табличном виде.
\end{itemize}

Снимок экрана с рабочим окном BitScope Logic представлен в приложении~\ref{app:figures} на рисунке~\ref{fig:bitscopelogic}.

\subsection{Выводы}

Рассмотренные комплексы предлагают несколько базовых подходов к представлению данных об обменах:

\begin{itemize}
 \item представление на уровне логических уровней во времени;
 \item представление в виде потока байтов и управляющих символов;
 \item табличное представление ``сырых'' данных обменов;
 \item табличное представление с расшифровкой данных обменов.
\end{itemize}

Каждое из представлений полезно в своих определённых задачах. Тем не менее, есть ограничения по использованию некоторых из них в задаче, поставленной в этой курсовой работе. Для того, чтобы определить требования к реализации поддержки шины I2C в среде Opermon, следует провести обзор уже существующего функционала Opermon.
