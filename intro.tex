\section{Введение}

При разработке и отладке встраиваемых систем очень часто возникает необходимость мониторинга, регистрации, визуализации и анализа данных, передаваемых по шинам данных устройства. У возникающих ошибок зачастую может быть разная природа, начиная от неправильной конфигурации приёмо-передающего блока и заканчивая ошибками в логике работы пользовательского программного обеспечения.

Возможностей традиционных отладчиков, предназначенных в первую очередь для поиска ошибок в ПО, в этом случае часто не хватает, так как в работе шины задействовано сразу несколько устройств, синхронизированных по времени. Более того, традиционный отладчик не даёт понимания того, что происходит на шине данных (невозможно или неоправданно сложно отследить состояние передачи и синхронизацию линий обмена данными). Поэтому для решения подобных задач используются специальные программно-аппаратные комплексы для мониторинга и анализа обменов.

Как правило, такой программно-аппаратный комплекс состоит из двух компонентов: адаптер для подключения персонального компьютера (ПК) к анализируемой шине данных и специального программного обеспечения (ПО) для обработки и визуализации данных.

В ЛВК ведется ряд работ, целью которых является создание инструментальной среды для мониторинга, регистрации, визуализации и анализа информационного обмена в различных бортовых авиационных каналах (ARINC 429, MIL STD 1553B, Fibre channel и др.)

Рабочей группой достигнуты значимые результаты и получен богатый опыт в данной области. Однако данные работы ведутся в интересах конкретных заказчиков и их результаты являются закрытыми разработками. С целью популяризации ЛВК в научном и техническом сообществе может иметь смысл создание на основе существующих разработок свободного инструмента для анализа информационного обмена.

С другой стороны, в области встроенных вычислительных систем, исследованием которых также занимаются в ЛВК, есть много специализированных интерфейсов, для которых инструменты анализа информационного обмена либо совсем отсутствуют, либо существуют, но проигрывают по возможностям аналогичным инструментам, разрабатываемым в ЛВК.

В качестве одного из таких интерфейсов можно рассмотреть шину I2C. Этот интерфейс широко используется для связи интегральных схем отдельных узлов встраиваемых вычислительных систем за счёт своей простоты и большого количества поддерживаемых устройств различного назначения. Однако, на сегодняшний день не существует свободного программно-аппаратного комплекса для мониторинга этой шины, ориентированного на анализ обменов на уровне сообщений. Основное множество доступных программных средств для мониторинга шины I2C ориентировано на анализ передач на физическом и канальном уровне.

\section{Цель и задачи курсовой работы}

Целью курсовой работы является адаптация существующей инструментальной среды Opermon для работы с адаптером шины I2C [\ref{i2c_protocol_spec}] и для визуализации  получаемых обменов с учётом специфики устройства шины.

Для достижения поставленной цели необходимо решить следующие задачи:

\begin{itemize}
 \item Провести обзор существующих программных и аппаратных средств анализа шины I2C (а также других специализированных интерфейсов встроенных вычислительных систем) с точки зрения:
 
 \begin{itemize}
  \item возможностей для мониторинга и анализа передач;
  \item организации пользовательского интерфейса;
  \item доступности для разработчика.
 \end{itemize}
 
 \item Выполнить их сравнение с разрабатываемой в ЛВК средой OperMon.
 
 \item Предложить проект реализации инструмента анализа информационного обмена по шине I2C на основе среды OperMon:
 
 \begin{itemize}
  \item выбрать аппаратный интерфейс к шине;
  \item спроектировать архитектуру средства;
  \item предложить набор представлений информации об обмене и спроектировать пользовательский интерфейс.
 \end{itemize}

 
 \item Реализовать инструмент анализа;
 
 \item Провести анализ существующей архитектуры среды Opermon и предложить проект рефакторинга с целью:
 
 \begin{itemize}
  \item более явного выделения интерфейсо-специфичной части;
  \item реализации возможности вариативной сборки с заданием списка поддерживаемых в данном варианте интерфейсов;
  \item разбиения исходного кода и собираемых объектных компонентов на независимые модули и библиотеки с возможностью выделения части этих модулей, достаточной для сборки средства мониторинга шины I2C
 \end{itemize}
 
\end{itemize}
