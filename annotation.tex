\section*{Аннотация}

Данная работа посвящена задаче создания инструмента для анализа и мониторинга информационного обмена для интерфейса I2C, часто применяемого для обмена данными между узлами встраиваемых вычислительных систем.

Раздел \ref{i2c_bus} содержит описание шины I2C, выбранной при постановке задачи данной работы в качестве целевой шины для реализации инструмента анализа.

В разделах \ref{i2c_review} и \ref{opermon} проведено исследование существующих на рынке свободных и коммерческих программно-аппаратных комплексов для анализа и мониторинга шины I2C, выявлены их особенности и сформулированы требования к разрабатываемому средству.

В разделах \ref{implementation} и \ref{testing} приведен проект доработки разрабатываемой в ЛВК инструментальной среды Opermon и описана реализация. Проведена апробация, сделаны выводы об удобстве и применимости, сформулированы замечания и предложения по доработке.

Также ряд предложений по доработке содержится в разделе \ref{refactoring}, посвященном задаче рефакторинга, основной целью которой является разделение
инструмента на базовую часть и компоненты, ответственные за реализацию поддержки конкретных интерфейсов. Данное разделение необходимо для опубликования анализатора I2C в открытом доступе.
